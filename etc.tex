利用規約
第 1 条(目的)
本規約は、株式会社 Second style(以下「当社」といいます。)が提供する LOSSCAN(以下「本
サービス」といいます。)の利用条件を定めるものです。利用者(以下「会員」といいます。)は、
本規約に同意のうえ本サービスを利用するものとします。
第 2 条(サービス内容)
本サービスは、顧客来店履歴データを基に AI 分析を行い、来店遅延傾向および継続リスクを可
視化する顧客ヘルス・マネジメント SaaS です。
本サービスは業務改善支援ツールであり、成果・売上向上・来店回復等を保証するものではあり
ません。
第 3 条(データの帰属と処理)
会員が登録・連携した顧客データの所有権は会員に帰属します。
当社は当該データを「受託処理者」として取り扱います。
データは当社管理サーバーに保存され、1 ヶ月保管後に削除されます。
解約時、CSV 形式でエクスポート可能です(期限内のみ)。
第 4 条(AI 学習)
個別顧客単位で AI 解析を行います。
匿名加工した統計データはモデル改善に利用します。
学習利用の拒否はメール申請により可能です。
拒否後は将来分の学習利用を停止します。
第 5 条(未払い)
支払期限経過時は即時停止。
7 日以内の支払確認で復旧。
7 日経過後は契約解除およびデータ削除対象となります。
第 6 条(禁止事項)
・不正アクセス
・再販売
・リバースエンジニアリング
・データの不正取得
・反社会的勢力の関与
・違法利用
・当社裁量で不適切と判断する行為
第 7 条(API 免責)
外部 API 障害、仕様変更、第三者サービス停止に起因する不具合について当社は責任を負いませ
ん。
データ整合性の完全保証は行いません。
第 8 条(稼働)
原則常時稼働。
メンテナンスは事前告知の上実施。
第 9 条(情報漏えい時)
漏えい発覚から 2 営業日以内に通知。
原因報告および再発防止策提示を行います。
第 10 条(解除)
軽微違反:警告
再違反:即解除
悪質行為:即時解除
損害が発生した場合は請求可能。
第 11 条(損害賠償)
当社の賠償上限は直近 1 ヶ月分の利用料。
逸失利益・特別損害は除外。
故意または重過失を除き適用。
第 12 条(反社会的勢力)
会員は反社会的勢力に該当しないことを表明保証します。
第 13 条(準拠法・管轄)
日本法準拠
東京地方裁判所を専属的合意管轄裁判所とします。



プライバシーポリシー
株式会社 Second style (以下「当社」といいます。)は、当社が提供する LOSSCAN 会社情報(以
下「本サービス」といいます。)において、利用者の個人情報を以下の方針で取り扱います。
1. 基本方針
株式会社 Second style は、個人情報保護法を遵守し、適正に取り扱います。
2. 取得情報
・氏名
・法人名
・メールアドレス
・電話番号
・IP アドレス
・顧客データ(受託処理対象)
3. 取得経路
・申込フォーム
・問い合わせフォーム
・会員ページ
・自動ログ取得
4. 利用目的
・サービス提供
・AI 分析
・モデル改善(匿名加工)
・システム改善
・障害対応
5. 保存期間
問い合わせ情報:半年
会員情報:半年
ログ情報:半年
バックアップ:1 ヶ月
削除依頼があった場合は対応可能。
6. 第三者提供
原則なし。
法令に基づく場合を除く。
7. 安全管理措置
・アクセス制限
・通信暗号化
・権限管理
・ログ監視
・定期的なセキュリティ見直し
8. AI 学習拒否
メール申請により学習利用停止可能。
9. 情報漏えい時対応
2 営業日以内通知
再発防止策提示
10. 改定
改定時は事前告知


特定商取引法に基づく表記

■ 事業者情報
事業者名
株式会社 Second style
代表者名
井上 太陽
運営責任者
代表取締役 井上 太陽
所在地
〒150-0044
東京都渋谷区円山町 5 番 3 号 MIEUX 渋谷ビル 8 階
電話番号
050-1725-4790
※電話によるサポートは原則行っておりません。
メールアドレス
info@secondstyle-inc.net
問い合わせ対応方法
原則メール対応
対応時間
平日 9:00〜17:00(土日祝休み)
営業時間外の問い合わせは翌営業日対応
緊急対応は行っておりません。
■ サービス名称
LOSSCAN
■ サービスの種類
クラウド型 顧客ヘルス・マネジメント SaaS
(ブラウザ上で利用可能な月額制データ分析ツール)
■ 販売価格
各プランごとに税込価格を明示しております。
詳細は本ページ内料金表をご参照ください。
■ 商品代金以外の必要料金
消費税
インターネット接続に必要な通信費
外部 API 連携時に発生する第三者サービス利用料(発生する場合)
※外部サービス利用料は各サービス提供事業者の定める料金体系に従います。
■ 申込みの成立時期
本サービスの利用契約は、
利用者による申込み完了および当社による決済確認が完了した時点で成立します。
■ 支払方法
クレジットカード決済のみ
■ 支払時期
初回:申込時に即時決済
以降:初回決済日を基準とした毎月自動更新
■ 役務の提供時期
初回決済完了後、直ちにアカウント発行を行い利用可能となります。
■ 提供期間
本サービスは月額制サービスです。
契約期間は 1 ヶ月単位で自動更新されます。
■ 動作環境
本サービスは以下の環境での利用を推奨します。
Google Chrome 最新版
Safari 最新版
JavaScript 有効環境
安定したインターネット接続
※外部 API 連携を行う場合は、当該サービスの動作環境に準じます。
■ 提供内容の範囲
本サービスは、顧客来店履歴データを基に来店周期・継続傾向を可視化するツールです。
以下は提供範囲外となります:
集客保証
売上保証
来店回復保証
経営判断の代行
法務・税務・会計アドバイス
本サービスは意思決定支援ツールであり、最終判断は利用者の責任において行うものとします。
■ 解約について
解約は次回決済日の 7 日前までに、会員ページよりメール申請にて行うものとします。
期限を過ぎた場合は次回分の決済が実行されます。
■ 中途解約
月額サービスの性質上、
日割り返金は行いません。
■ 返金について
デジタルサービスの特性上、原則返金は行いません。
ただし、以下の場合に限り返金を行う場合があります:
当社の重大な過失によりサービス提供が不可能となった場合
明示された機能が提供されなかった場合
■ クーリングオフについて
本サービスは通信販売によるデジタル役務提供のため、
特定商取引法に基づくクーリングオフの対象外となります。
■ アカウント管理責任
利用者はアカウント情報の管理責任を負います。
第三者による不正利用が発生した場合も、利用者の管理責任とみなされる場合があります。
■ 未払い時の対応
決済失敗時は即時アカウント停止となります。
7 日以内に支払いが確認できた場合は復旧可能です。
7 日を超過した場合は自動解約となります。
■ サービス停止・変更
以下の場合、サービスを停止・変更することがあります。
システム保守
セキュリティ対応
法令改正
外部 API 仕様変更
変更時は原則 1 週間前に告知します。
■ 稼働保証
原則として安定稼働を目指しますが、
メンテナンス等により一時停止する場合があります。
メンテナンス実施時は事前告知します。
■ 情報漏えい時の対応
個人情報漏えいが判明した場合、
2 営業日以内に対象者へ通知し、再発防止策を提示します。
■ 損害賠償責任の上限
当社の責任は、直近 1 ヶ月分の利用料金を上限とします。
逸失利益・特別損害は除外します。
ただし、当社の故意または重過失による場合はこの限りではありません。
■ 反社会的勢力の排除
反社会的勢力に該当する場合、即時契約解除を行います。
■ 準拠法・管轄裁判所
本表示に関する紛争は日本法を準拠法とし、
東京地方裁判所を第一審の専属的合意管轄裁判所とします。