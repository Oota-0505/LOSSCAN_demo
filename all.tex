# **LP仕様書_LOSSCAN**

## **0. LP概要(Summary)**

**■サービスの要点**

LOSSCANは、**すでに一定数の来店実績がある店舗**を対象に、来店周期・継続傾向をAIで学習し、

「来なくなりそうな顧客」だけを自動で抽出・可視化する顧客ヘルス・マネジメントSaaS。

**■このLPの役割**

- 新規集客ツールではないことを明確に伝える
- 「継続率改善」という一点に絞った専門サービスだと理解させる
- 1980円から始められる“現実的な導入ハードル”を提示する

**■提供価値**

- 感覚・経験・勘に依存していたフォロー判断を
    
    → データに基づく“優先順位のある行動”に変える
    
- 防げたはずの継続ロスを、構造的に減らす

**■想定ターゲット**

- 来店履歴データがすでに蓄積されている
- 小〜中規模の予約型店舗・スクール事業者

**■導入前提(重要)**

LOSSCANは「新規が来る仕組み」を作るツールではなく、
すでに来店実績がある店舗の“継続ロス”を減らすための仕組みです。

そのため、以下のような状態の店舗ほど効果を実感しやすい設計です。

- 来店/予約履歴が一定数たまっている(例:半年以上の履歴がある)
- 顧客数が増えてきて、手作業で追いきれなくなっている
- フォローの基準が人によってブレている

※「必ず離脱する/必ず戻る」を断定するものではなく、
あくまで“来店が遅れ始めた兆候”を見つけ、行動の優先度を整えるためのサービスです。

---

## **1. サービス内容(Service Overview)**

### ■ サービス概要

LOSSCANは、

「来店が止まってから気づく」状態をなくすためのツールです。

予約/来店データをもとに、

顧客ごとの通常来店周期を自動算出。

その周期からズレ始めた顧客だけを検知し、

**“今フォローすべき人”を明確にします。**

売上を伸ばすための新規施策ではなく、

**すでに積み上がっている売上を守るための仕組み**として設計されています。

※連携データについて
LOSSCANは、予約/来店履歴(日時・メニュー・担当など)をもとに、
「その人の通常ペース」と「遅れ始めた兆候」を整理します。

連携するシステムやデータ項目は店舗ごとに異なるため、
初期設定時に「取り込める範囲」と「できること/できないこと」を明確にしたうえで開始します。

### ■ プラン構成(全体像)

- Lite Plan:継続状況の可視化(入口)
- Solo Plan:フォロー漏れ防止の仕組み化
- Studio Plan:分析による改善フェーズ
- Multi-location Plan:経営判断レベルの可視化

### ■ オプション

- 初期設定費(API接続/データ整備)
- 追加システム接続
- 月次レポート拡張(KPI簡易モニタリング)

### ■ ターゲットユーザー像(具体)

- 美容室・整体・ネイル・リラク・パーソナルジム
- 英会話・音楽・習い事などのスクール型ビジネス
- 来店履歴はあるが、活用しきれていない事業者

### ■ 特徴(詳細)

1. **顧客単位で来店ペースを算出**
    
    平均値ではなく「その人基準」で“遅れ”を捉えます。
    
2. **来店遅延の兆候を早期に検知**
    
    完全離脱前に気づける仕組み。
    
3. **行動につながる優先度表示**
    
    RED/YELLOWを見るだけで判断可能。
    
4. **構造的な傾向分析**
    
    スタッフ・メニュー・店舗単位で原因が分かる。
    
5. **現場負担を増やさない設計**
    
    見る → 判断 → 動く、だけ。
    

---

## **2. トーン&マナー(Tone & Manner)**

**■世界観**

冷静・論理的だが、現場目線でわかりやすい。

**■ブランドイメージ**

「攻め」ではなく「守り」で利益を積み上げる、堅実なパートナー。

**■雰囲気キーワード**

- 落ち着き
- 信頼感
- データドリブン
- 現実的
- 無理をさせない

**■参考トーン**

SaaS系BtoBサービス(CRM/分析ツール系)

---

## **3. カラーガイド(Color Palette)**

- メインカラー:#1E293B(ダークネイビー)→ 信頼・安定・分析ツール感
- サブカラー:#F8FAFC(ライトグレー)→ 情報の可読性を高める
- アクセントカラー:#F59E0B(アンバー)→ 注意・優先度・RED/YELLOWの視認性

**■使用意図**

感情を煽らず、状況を「判断」させる配色。

---

## **4. フォントガイド(Typography)**

- 見出し:Noto Sans JP(Medium〜Bold)
- 本文:Noto Sans JP(Regular)
- 英字:Inter

**■使い方**

- 見出しは簡潔に
- 本文は行間広めでストレスを減らす
- 数値・価格は太字で強調

---

## **5. レイアウト(Layout Guideline)**

- 1カラム基本
- 重要箇所のみ2カラム
- 余白多め
- 角丸は弱め
- 影は極小
- グラデーションは使わない

---

## **6. 写真・ビジュアル(Visual Guideline)**

**■人物写真**

- 実店舗オーナー
- PC/タブレットを見る姿
- 作り込みすぎないリアル感

**■色味**

- 彩度低め
- 暖色寄り

**■NG**

- 成果誇張
- 派手な演出
- ストック感強すぎる写真

---

## **7. コピー(Copy Guideline)**

**■トーン**

断定しすぎない/煽らない/現実的。

**■禁止表現**

- 必ず売上UP
- 誰でも簡単
- 放置でOK
- 〇日以内に離脱すると分かる
- 未来を正確に予測する
- 絶対に失客を防げる

**※冒頭コピー案**

> そのお客様、
> 
> 
> 「もう来ない」のではなく
> 
> 「声をかける前だった」だけかもしれません。
> 

---

## **【1. ヒーローセクション】**

### ■ 目的

- 「新規集客ツールではない」ことを即座に理解させる
- *LOSSCAN = 継続率のための専門ツール* と認知を固定する
- Lite(1,980円)への心理的ハードルを下げる

---

### ■ タイトル(H1)

```
来なくなりそうな顧客を、
来なくなる前に。
```

※2行固定推奨

※改行位置はデザインで調整可だが、意味分断しない

### ■ リードコピー(H2〜P)

```
LOSSCANは、新規集客を増やすツールではありません。
すでに来ている顧客を、静かに失わないための仕組みです。

感覚や勘に頼っていたフォロー判断を、
データに基づく「今、声をかけるべき人」へ。
```

※1文ずつ改行

※語尾を断定しすぎない

※「AI」「分析」という言葉はここでは控えめに

### ■ 要点(バレット・3点)

```
・来店遅延の兆候を自動で検知
・フォロー優先度をRED/YELLOWで可視化
・現場は“見るだけ”で行動できる
```

※アイコン付き推奨

※数を増やさない(最大3つ)

### ■ CTA(ヒーロー内)

- メインCTA:
**「月額1,980円で、継続状況をまず見てみる(Liteプラン)」**
- サブCTA:
「資料を請求する」

※メインCTAの直下に補足(小さめ)
「新規集客ツールではありません。既存顧客の継続に特化した仕組みです。」

### ■ デザイン指示

- レイアウト:
    
    左テキスト/右ビジュアル(PC)
    
    SPでは縦積み
    
- ビジュアル:
    
    ダッシュボードUIの一部(RED/YELLOWリスト)+
    
    PC/タブレットを見る店舗オーナー写真
    
- 背景:
    
    薄いグレー or ホワイト
    
- 注意:
    
    KPIグラフを前面に出しすぎない(難しく見えるため)
    

---

## **【共感セクション】**

### ■ 目的

- 「これは自分の店の話だ」と思わせる
- 課題を“努力不足”ではなく“構造問題”として描く

---

### ■ セクションタイトル

```
こんな状態に、心当たりはありませんか?
```

---

### ■ 悩みリスト(チェック形式)

```
□ 気づいたら、来なくなっていたお客様がいる
□ 誰に声をかけるべきか、スタッフごとに判断が違う
□ 忙しくて、全顧客を見きれない
□ 失ってから「もったいなかった」と感じることが多い
```

---

### ■ 共感ストーリー(短文)

```
多くの店舗で起きているのは、
「フォローしなかった」のではなく、
「気づけなかった」という問題です。

予約システムは、
“来る予定の人”は教えてくれても、
“来なくなり始めた人”までは教えてくれません。

だからこそ、失ってから動くのではなく、
“遅れ始めた段階で気づける状態”を先に作ることが、
継続率改善の近道になります。
```

### ■ デザイン指示

- 背景:薄いサブカラー
- テキスト幅:狭め(読みやすさ優先)
- イラスト or 写真:
    
    悩んでいるオーナー1点のみ(入れすぎない)
    

---

# **【LOSSCANが選ばれる理由】**

### ■ 目的

- 「なぜ他ではなくLOSSCANなのか」を論理で説明
- 無料AI・予約システムとの差を明確化

---

### ■ セクションタイトル

```
LOSSCANが選ばれる理由
```

### ■ 理由1|顧客ごとの来店周期を学習

**見出し**

```
平均ではなく、「その人のリズム」を見る
```

**本文**

```
LOSSCANは、全体平均や一律ルールでは判断しません。
過去の来店履歴から、その顧客ごとの通常周期を算出し、
ズレ始めたタイミングを検知します。
```

### ■ 理由2|行動につながる優先度設計

**見出し**

```
判断ではなく、「行動」を助ける設計
```

**本文**

```
分析結果を読み解く必要はありません。
RED/YELLOWを見るだけで、
今、誰に声をかけるべきかが分かります。
```

### ■ 理由3|新規集客に寄らない専門性

**見出し**

```
「増やす」より、「失わない」に集中
```

**本文**

```
LOSSCANは、新規集客機能をあえて持ちません。
既存顧客の継続に特化することで、
現場で本当に使われる仕組みに絞っています。
```

### ■ 補足|予約システム/無料AIとの違い

- 予約システム:来店予定は分かっても、「来なくなり始めた兆候」は拾いにくい
- 無料AI:文章や一般論は作れても、「自店舗データの来店リズム」を前提にした判断は難しい

LOSSCANは、来店履歴という“自店舗の事実データ”に立脚して、
「今フォローすべき順番」を整えることに集中しています。

### ■ デザイン指示

- 3カラム(SPは縦)
- アイコン+テキスト
- 数字・グラフはここでは使わない

---

## **【利用者の声】**

### ■ 目的

- 導入後の「変化」を具体的に想像させる
- 売上数値より“判断の変化”を語らせる

### ■ 声①|美容室(1店舗)

```
正直、来なくなってから気づくことが多かったです。
LOSSCANを入れてからは、
「今、声をかけるべき人」が明確になりました。
```

### ■ 声②|整体院(3店舗)

```
感覚ではなく、共通の基準で話せるようになったのが大きいです。
スタッフへの指示もスムーズになりました。
```

### ■ 声③|パーソナルジム

```
失ってから後悔することが減りました。
静かに効いている感覚があります。
```

### ■ デザイン指示

- 顔写真あり(小さめ)
- 吹き出し or カード
- コメントは短め

---

## **【サービス内容一覧】**

### ■ 目的

- 機能漏れ確認
- 比較検討ユーザーへの安心感

### ■ 内容

```
・顧客ごとの来店周期学習
・来店遅延の自動検知
・フォロー優先度表示(RED/YELLOW/GREEN)
・スタッフ別/メニュー別の継続傾向分析
・CSVエクスポート
・リスト更新(プランに応じた頻度)
・判定基準(RED/YELLOW)の初期設定・見直し(プランにより対応範囲が異なります)
```

---

### ■ デザイン指示

- チェックリスト
- 2カラム可

## **【ご利用の流れ】**

### ■ 目的

- 導入の難しさを感じさせない
- SaaSに不慣れな層への配慮

### ■ STEPテキスト

```
STEP1|お申し込み
STEP2|初期設定(データ連携)
※予約/来店履歴の取り込み範囲を確認し、判定に必要な項目を整えます
STEP3|自動分析スタート
STEP4|フォロー対象リスト確認
STEP5|優先顧客へアクション
STEP6|継続状況を定期チェック
```

### ■ デザイン指示

- 横ステップ(SPは縦)
- アイコン簡素

---

## **【料金プラン】**

## ● Lite Plan(ライトプラン)|月額 1,980円(税込2,178円)

**まずは“継続の見える化”から始めたい方向け**

```
・顧客数:〜500名まで
・接続可能サービス:1システム
・来店周期の自動推定(簡易)
・来店遅延顧客の一覧表示(YELLOW/RED)
・フォロー対象CSVエクスポート
```

※継続状況の把握を目的としたエントリープラン

※分析レポート・優先度自動ソートは含まれません

※Liteは「まず継続の遅れを見える化する」ことを優先した設計です。
精密な優先度ソートや詳細分析よりも、導入の軽さを重視しています。

※Lite=「遅れ始めを見つける入口」、Solo=「優先度まで整えて迷いを減らす基本」と考えると分かりやすいです。

## ● Solo Plan(ソロプラン)|月額 3,980円(税込4,378円)

**フォロー漏れを“仕組みで防ぎたい”店舗向け**

```
・顧客数:〜1,000名まで
・接続可能サービス:1システム
・来店遅延アラート更新:週1回
・顧客ごとの来店周期の自動推定
・危険度別(GREEN/YELLOW/RED)のリスト表示
・CSVエクスポート機能
```

※「誰に声をかけるべきか」を迷わない状態を作るプラン

## ● Studio Plan(スタジオプラン)|月額 6,980円(税込7,678円)

**継続率を“分析で改善したい”事業者向け**

```
・顧客数:〜5,000名まで
・接続可能サービス:2システムまで
・来店遅延アラート更新:週2回
・メニュー別/スタッフ別の継続傾向レポート
・新規→2回目/2回目→3回目の定着率分析
・フォロー優先度の自動ソート(重要度順)
```

※改善ポイントを構造的に把握したい店舗向け

## ● Multi-location Plan(マルチロケーションプラン)|月額 12,800円(税込14,080円)

**多店舗経営を“数字で安定させたい”方向け**

```
・顧客数:〜10,000名まで
・接続可能サービス:3システムまで
・店舗別ダッシュボード
・来店遅延リスクの高い店舗ランキング
・役員/マネージャー向け月次レポートPDF自動生成
・優先サポート(初期設定・API接続)
```

※現場管理ではなく、経営判断レイヤー向けのプラン

## ● オプション

```
・初期設定費:9,800円(税込10,780円)〜19,800円(税込21,780円)
 (API接続/データ整備)
※料金幅は、連携システム数・データ整備量(項目の欠け/統合状況)により変動します。

・追加接続:1サービスごとに
 1,000円(税込1,100円)〜3,000円(税込3,300円)/月

・月次レポート拡張:2,000円(税込2,200円)/月
 (改善ポイント+主要KPIの簡易モニタリング)
```

※これにより、ARPUとLTVを安定化させ、

小規模市場でも持続可能な収益モデルを構築できます。

## デザイン指示(料金セクション)

- **Liteを一段目で強調**(枠線・背景色)
- Solo / Studio / Multi は横並び
- 各プランの「書き出し文」は必ず視認できるサイズで
- 「おすすめ」ラベルは付けない(売り込み感を避ける)

---

## **【CTA】**

### ■ コピー

```
まずは、
「来なくなりそうな顧客」を知るところから。
```

### ■ ボタン

- メイン:Liteプランを見る(1,980円)
- サブ:ダッシュボードの画面イメージを見る
- 予備:資料請求

### ■ デザイン指示

- 背景:落ち着いた店舗写真
- 余白広め
- CTAは1画面に1回

---

## **【よくある質問(FAQ)】**

**Q. 新規集客にも使えますか?**

A. 目的は新規獲得ではなく、既存顧客の継続率改善です。来店が遅れ始めた兆候を見つけ、フォローの優先度を整えます。

**Q. どんなデータが必要ですか?**

A. 予約/来店履歴(日時・顧客IDなど)が必要です。取り込める項目は連携システムにより異なるため、初期設定で確認します。

**Q. 「RED/YELLOW」は何を基準に決まりますか?**

A. 直近の来店履歴から、その顧客の通常ペースに対して“遅れ始めた度合い”を見て整理します。基準の考え方は初期設定で共有します。

**Q. 必ず失客を防げますか?**

A. 失客や売上の保証は行いません。あくまで「気づける状態」を作り、行動の優先順位を整えるための仕組みです。

**Q. 途中解約はできますか?**

A. はい。次回更新前までに手続きいただければ停止できます。

**Q. 返金はできますか?**

A. サービスの性質上、決済後の返金は原則行っていません。不安がある場合はLiteからの導入をおすすめします。

**Q. どの予約システムに対応していますか?**

A. 連携可否はシステム仕様により異なります。対応状況は事前確認のうえ、取り込めるデータ範囲に合わせて開始します。まずは「資料請求」または「画面イメージ」からご確認ください。

**Q. データの取り扱いは安全ですか?**

A. 取り扱い方針は「データ取り扱いについて(セキュリティ)」に記載しています。必要に応じて、導入前に運用上の注意点も共有します。

---

## **【会社概要】**

・会社名:〇〇〇〇(※正式名称を記載)

・代表者:〇〇 〇〇

・所在地:〇〇県〇〇市(※詳細住所は特商法表記に記載)

・事業内容:サロン/スクール向けデータ活用支援、顧客分析・継続率改善を目的としたSaaSの企画・提供、Web/マーケティング支援、コンテンツ企画・制作

※Googleマップ埋める

---

## **【フッター】**

・LOSSCANとは
・サービスの特徴
・料金プラン
・導入までの流れ
・よくある質問(FAQ)

・特定商取引法に基づく表記
・プライバシーポリシー
・利用規約
・お問い合わせ

・データ取り扱いについて(セキュリティ)

**↓ ↓ ↓ 会員登録&マイページ仕様案内 ↓ ↓ ↓**

---

# **■ 会員登録・マイページ機能 仕様案内(LOSSCAN)**

---

## **■ 本サービスで利用する主要機能**

本サービスでは、会員が以下の情報を確認・利用できる

**簡易マイページ**を提供します。

- 契約プランの確認
- 請求・支払い状況の確認
- 利用ガイド・資料のダウンロード
- サポートへの問い合わせ

※決済サービスとのAPI連携は行わず、

**運営側での手動管理・反映を前提としたシンプルな構成**とします。

---

## **■ アカウント情報**

**表示項目**

- 氏名
- メールアドレス(ログインID)
- 電話番号(任意)
- 住所(必要時のみ)
- パスワード変更

**仕様**

- 会員本人が編集可能
- パスワードはマスク表示
- メールアドレス変更時は確認メールを送付(任意)

---

## **■ 申込履歴**

**表示項目**

- 申込日
- プラン名
- 決済金額
- 決済方法
- ステータス

**仕様**

- 決済完了メールを基に、運営が手動で登録
- 会員側は閲覧のみ(編集不可)
- APIによる自動反映は行わない

---

## **■ 契約ステータス表示**

**表示例**

- 受付済
- 決済確認済
- 提供中
- 解約申請中
- 解約済

**仕様**

- ステータスは運営側が手動で更新
- 会員は現在の状態を確認するのみ
- 自動切替・自動更新は行わない

---

## **■ 請求情報(手動管理)**

**表示項目**

- 今月の請求額
- 支払い履歴
- 次回更新日

**仕様**

- 請求情報は運営側が管理画面から手動入力
- クレジットカード番号等の決済情報は保持しない
- 決済処理・カード情報管理はPSP(決済代行サービス)側に一任

---

## **■ ダウンロード資料**

**表示内容**

- 利用ガイドPDF
- 初期設定マニュアル
- プラン特典PDF
- レポート/成果物ファイル(該当プランのみ)

**仕様**

- 運営が管理画面から手動でアップロード
- プラン別に表示・非表示の制御が可能
- 会員は閲覧・ダウンロードのみ

---

## **■ サポート機能**

### ● お問い合わせフォーム

**設置内容**

- 相談内容を複数選択できるチェックボックス形式

**選択項目例**

- サービス内容について
- プラン・契約・解約について
- データ連携・初期設定について
- レポート内容について
- その他

**仕様**

- チェック項目に応じて、関連入力フィールドを表示(任意)
- 問い合わせ内容は運営宛にメールまたは管理画面通知
- 会員マイページ内から常時アクセス可能

---

### ● チャットサポート(任意)

**内容**

- LINEチャットサポート(FLOWGRAM公式LINE)

**仕様**

- API連携は行わず、通常の外部リンクとして設置
- 対応時間・対応範囲は別途案内

---

## **■ 退会・アカウント削除**

**表示項目**

- 退会申請ボタン
- 注意書き(退会後のデータ取扱い等)

**仕様**

- ボタン押下で退会申請フォームへ遷移
- アカウント削除およびサブスクリプション停止は、
    
    運営側が内容を確認の上、手動で対応
    
- 自動即時削除は行わない

---

## **■ 共通UI・補足仕様**

- ヘッダーに検索フィールドを設置
    
    (ガイド・資料・FAQ検索用)
    
- マイページ全体は**機能過多にせず、確認・申請用途に限定**
- 操作ログの自動保存・解析等は行わない

---

## ✅ この仕様の意図(内部共有用メモ)

- **API連携なし=軽量・低コスト・PSP安全設計**
- Lite(1,980円)〜上位プランまで同一UIで運用可能
- 新規法人スタート時でも実装・運用が現実的
- LP・特商法・FAQとの整合が取りやすい